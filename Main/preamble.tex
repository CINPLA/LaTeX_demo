\documentclass[10pt, notitlepage]{article} %twoside, %to get name on one page and title on the other
\usepackage[a4paper, margin=2.7cm, twoside]{geometry} % , twoside
\usepackage[utf8]{inputenc}
%\usepackage{lipsum}
\usepackage{upgreek}
\usepackage{graphicx}
\usepackage{titlesec} %appendix
\usepackage{amsmath,array}
\usepackage{amsfonts}
\usepackage{amssymb}
\usepackage{amsthm}
\usepackage[table,dvipsnames]{xcolor}
\usepackage{hyperref}
\usepackage[titletoc,title]{appendix}
\usepackage{sectsty} %header
\usepackage{fancyhdr} %header
\usepackage{placeins} %FloatBarrier
\usepackage{floatrow} %caption side
\usepackage{subfig} %for subref 
\usepackage{caption}
\usepackage[round]{natbib} %bibliography

\usepackage{pdfpages}
\usepackage{todonotes} %[disable]
\usepackage{cleveref}% Has to be loaded after hyperref [nameinlink]

%-------------------------------------------------------------------------------
%                                 INCREASE LINESPACE
%-------------------------------------------------------------------------------

%\renewcommand{\baselinestretch}{1.5}

%-------------------------------------------------------------------------------
%                                 REFCHECK
%-------------------------------------------------------------------------------
%\usepackage{refcheck} %reference check
%%%% Infrastructure    
%\makeatletter
%\newcommand{\refcheckize}[1]{%
%  \expandafter\let\csname @@\string#1\endcsname#1%
%  \expandafter\DeclareRobustCommand\csname relax\string#1\endcsname[1]{%
%    \csname @@\string#1\endcsname{##1}\wrtusdrf{##1}}%
%  \expandafter\let\expandafter#1\csname relax\string#1\endcsname
%}
%\makeatother
%%%%
%
%%%% Now we add the reference commands we want refcheck to be aware of
%\refcheckize{\cref}
%\refcheckize{\Cref}
%-------------------------------------------------------------------------------
%                                 DIV
%-------------------------------------------------------------------------------
\usepackage{tabularx}
\setlength{\extrarowheight}{2pt} %increase spacing between rows in tables

%-------------------------------------------------------------------------------
%                                 HYPERREF
%-------------------------------------------------------------------------------
\definecolor{dkgreen}{rgb}{0,0.6,0}
\definecolor{gray}{rgb}{0.5,0.5,0.5}
\definecolor{mauve}{rgb}{0.58,0,0.82}

\hypersetup{
colorlinks,
citecolor=BlueViolet,
filecolor=Blue,
linkcolor=Blue,
urlcolor=Blue
}


%-------------------------------------------------------------------------------
%                                 TABLE ROW SPACE
%-------------------------------------------------------------------------------
\newcommand{\tabspace}{\\[3pt]}
%-------------------------------------------------------------------------------
%                                 SUBFIG
%-------------------------------------------------------------------------------
\captionsetup[subfigure]{
position=top,
singlelinecheck=0,
justification=raggedright,
font={sf,large},
format=hang,
labelfont=bf,
labelformat=simple,
listofformat=subparens,
subrefformat=parens,
}
\renewcommand\thesubfigure{\Alph{subfigure}}
%-------------------------------------------------------------------------------
%                                 CLEVERREF
%-------------------------------------------------------------------------------
\newcommand{\csub}[2]{Fig.~\ref{#1}\subref{#2}}
\newcommand{\Csub}[2]{Figure \ref{#1}\subref{#2}}
\newcommand{\csubs}[3]{figs. \ref{#1}\subref{#2} and \subref{#3}}
\newcommand{\Csubs}[3]{Figures \ref{#1}\subref{#2} and \subref{#3}}
\newcommand{\csubrange}[3]{figs. \ref{#1}\subref{#2} -- \subref{#3}}
\newcommand{\Csubrange}[3]{Figures \ref{#1}\subref{#2} -- \subref{#3}}
\newcommand{\crefrangeconjunction}{~--~}

\crefname{appendix}{Appendix}{appendices}
\crefname{section}{Section}{sections}
\crefname{table}{Table}{tables}
\Crefname{table}{Table}{Tables}
\crefname{algorithm}{Algorithm}{algorithms}
\Crefname{algorithm}{Algorithm}{Algorithms}
\crefname{theorem}{Theorem}{theorems}
\crefname{conjecture}{Conjecture}{conjectures}
\crefname{corollary}{Corollary}{corollaries}
\crefname{lemma}{Lemma}{lemmas}
\crefname{example}{Example}{examples}
\Crefname{example}{Example}{Examples}
\crefname{remark}{Remark}{remarks}
\crefname{definition}{Definition}{definitions}
\crefname{equation}{Eq.}{eqs.}
\Crefname{equation}{Equation}{Equations}
\crefname{figure}{Fig.}{figs.}
\Crefname{figure}{Figure}{Figures}
\crefname{subfigure}{Fig.}{figs.}
\Crefname{subfigure}{Figure}{igures}
%-------------------------------------------------------------------------------
%                                 FOOTNOTES
%-------------------------------------------------------------------------------
%\usepackage{endnotes}
%\let\footnote=\endnote % places footnotes where you write \theendnotes
%\makeatletter
%\newcommand\footnoteref[1]{\protected@xdef\@thefnmark{\ref{#1}}\@footnotemark}
%\makeatother
%-------------------------------------------------------------------------------
%                                 THEOREMS
%-------------------------------------------------------------------------------
\newtheorem{theorem}{Theorem}[section]
\newtheorem{lemma}{Lemma}[section]
\newtheorem{definition}{Definition}[section]
\newtheorem{corollary}{Corollary}[section]
\newtheorem{conjecture}{Conjecture}[section]

\theoremstyle{definition}
\newtheorem{exmp}{Example}[section]


\theoremstyle{remark}
\newtheorem{remark}{Remark}[section]
\newtheorem*{remark*}{Remark}
%-------------------------------------------------------------------------------
%                                 HEADER/FOOTER
%-------------------------------------------------------------------------------

\setlength{\headheight}{15.2pt}
\pagestyle{fancy}

\fancyhf{}

\fancyhead[LE]{\thepage~~~~\nouppercase{\rightmark}}
\fancyhead[RO]{\nouppercase{\rightmark}~~~~\thepage}
\fancyhead[LO]{\nouppercase{\leftmark}}
\fancyhead[RE]{\nouppercase{\leftmark}}
\fancyfoot{}
%\fancyfoot[LE,RO]{\thepage}

\renewcommand{\sectionmark}[1]{\markboth{#1}{}}
%-------------------------------------------------------------------------------
%                                 SECTIONSTUFF
%-------------------------------------------------------------------------------
\newcommand{\emptypage}{\newpage
\thispagestyle{empty}
\mbox{}}
%                                   twoside
%-------------------------------------------------------------------------------
%\setlength{\oddsidemargin}{5mm}
%\setlength{\evensidemargin}{-5mm}
%                                  open section newpage
%-------------------------------------------------------------------------------
\newcommand*\stdsection{}
\let\stdsection\section
\renewcommand*\section{%
\clearpage %\ifodd\value{page}\else\mbox{}\clearpage\fi %open right
\stdsection}

%                                 sectionrule
%-------------------------------------------------------------------------------
\sectionfont{
	\sectionrule{0pt}{0pt}{-5pt}{0.8pt}
}
\numberwithin{equation}{section}


%-------------------------------------------------------------------------------
%                                 DEFINITIONS
%-------------------------------------------------------------------------------

\newcommand*\oline[1]{%
  \vbox{%
    \hrule height 0.5pt%                  % Line above with certain width
    \kern0.25ex%                          % Distance between line and content
    \hbox{%
      \kern-0.1em%                        % Distance between content and left side of box, negative values for lines shorter than content
      \ifmmode#1\else\ensuremath{#1}\fi%  % The content, typeset in dependence of mode
      \kern-0.1em%                        % Distance between content and left side of box, negative values for lines shorter than content
    }% end of hbox
  }% end of vbox
}
\renewcommand{\tilde}{\widetilde}

\newcommand{\pderivative}[2]{{\frac{\partial {#1}}{\partial {#2}}}}
\newcommand{\pdderivative}[2]{\frac{\partial^2 {#1}}{\partial {#2}^2}}
\newcommand{\derivative}[2]{\frac{\mathrm{d} #1}{\mathrm{d} #2}}
\newcommand{\dderivative}[2]{\frac{\mathrm{d}^2 {#1}}{\mathrm{d} {#2}^2}}
\newcommand{\intdef}[4]{\int\limits_{#1}^{#2} #3\, \mathrm{d}{#4}}
\newcommand{\inta}[3]{\int\limits_{#1} #2\, \mathrm{d}{#3}}
\newcommand{\ex}[1]{\mathrm{e}^{#1}}
\newcommand{\tx}[1]{\text{#1}}
\newcommand{\veca}[1]{\boldsymbol{#1}}
\newcommand{\norm}[1]{\left|\left|{#1}\right|\right|}
\newcommand{\abs}[1]{\left|{#1}\right|}
\newcommand{\Parantesis}[1]{\bigg({#1}\bigg)}
\newcommand{\parantesis}[1]{\big({#1}\big)}
\newcommand{\red}[1]{{\textbf{\color{red}*#1*}}}
\newcommand{\operator}[1]{\langle{#1}\rangle}
\newcommand{\BBrackets}[1]{\bigg[{#1}\bigg]}
\newcommand{\Brackets}[1]{\big[{#1}\big]}
\newcommand{\brackets}[1]{[{#1}]}
\newcommand{\foralls}{\:\forall\:}